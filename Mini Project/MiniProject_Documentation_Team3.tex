\documentclass{article}

\title{SEED Lab Mini Project Team 3 Documentation}
\author{
    Dawson J. Gullickson
    \and
    Julia Kaase
    \and
    Nick Nijkamp
    \and
    Sam Leonard
}
\date{\today}

% \usepackage{amsmath}
\usepackage{geometry}
\usepackage{hyperref}
% \usepackage{graphicx}
% \usepackage{listings}
% \usepackage{multicol}
\usepackage{xcolor}

% Margins (geometry)
\geometry{margin=1in}

% % Code Styling (listings, xcolor)
% \definecolor{mygreen}{rgb}{0,0.6,0}
% \definecolor{mygray}{rgb}{0.5,0.5,0.5}
% \definecolor{mymauve}{rgb}{0.58,0,0.82}
% \lstset{
%     basicstyle=\footnotesize,
%     commentstyle=\color{mygreen},
%     keywordstyle=\color{blue},
%     stringstyle=\color{mymauve},
%     numbers=left,
%     breaklines=true,
%     showstringspaces=false,
% }

% No page numbers
\pagenumbering{gobble}

\begin{document}
    \maketitle
	
    \begin{enumerate}
        % 1
        \item 
        
        % 2
        \item 
            [insert controlDesign.png]
        Below is the closed loop step response graph generated from the motor.
            [insert 2-closedLoop.png]
        Here is the experimental data generated from the closed loop PI controller:
            [insert closed loop experimental - Dawson has the data?]
        The closed loop resoponse adjusts well for overshoot and undershoot. The feedback allows for the motor to maks adjustments as needed. 

        Below is the open loop step response graph generated from the motor.
            [insert 2-openLoop.png]
        Implementing the open loop response would not work well, the motor goes off to infinity, and does not keep track of where it should be. 
        % 3
        \item 
        Arduino Code:

            PID Variables:
            double Kp = 15.3786175942488; [units - Dawson has them?]
            double Ki = 2.37803426483209; [units - Dawson has them?] 
            double Kd = 0; [units - Dawson has them?]
            Controller Code:
                    // calc error
                double e = r-y; // find where it needs to move from where it is

                if (Ts > 0) {
                    D = (e-e_past)/Ts; // derivative
                    e_past = e; // update val to get other vals
                } else {
                    D = 0;
                }

                I = I+Ts*e; // integral

                // Calc controller output -- output voltage uses PID
                double u = Kp*e+Ki*I+Kd*D;
                // deals with actuator saturation
                // i.e. if trying to write a voltage too high for board to supply
                if (abs(u) > umax) {
                    u = sgn(u)*umax;
                    e = sgn(e)*min(umax/Kp, abs(e));
                    I = (u-Kp*e-Kd*D)/Ki; 
                }
                
                // Convert voltage to speed
                int speed = -u*400/umax;

            [insert controlDesign.png]

        % 4
        \item Github repository: \url{https://github.com/nrnijkamp/SEEDLab}
    \end{enumerate}
\end{document}